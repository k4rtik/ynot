\section{Specification Constructs}

Jahob introduces the following specification constructs:
\begin{enumerate}
\item procedure constracts, consisting of:
      \begin{itemize}
      \item \q{requires} clause (precondition)
      \item \q{modifies} clause (set of modified locations)
      \item \q{ensures} clause (postcondition, potentially as a function of the precondition).
      \end{itemize}
\item procedure invariants, such as loop invariants
\item specification variables, introduced by $\q{specvar}$ declaration, which can be
      \begin{itemize}
      \item static specification variables (when their declaration contains $\q{static}$ keyword)
      \item instance specification variables
      \end{itemize}
      A specification variable can be:
      \begin{itemize}
      \item \emph{public}, used in public specification of the class, or
      \item \emph{private}, used to describe properties within
      the class implementation.  
      \end{itemize}
      In addition, concrete variable of class $C$ 
      can be marked as \emph{claimed} by a different class $D$,
      which makes them effectively variables private to class $D$ although they
      are physically stored in $C$ and accessed through instances of $C$.

      According to how they depend other variables specification variables can be
      \begin{itemize}
      \item dependent: given in terms of other variables; we distinguish three uses of dependent
            variables
            \begin{itemize}
            \item specifying abstraction function: specifying public specification variables
                  in terms of implementation variables (using private vardef construct
	          to define the contents of public specification variables); this is the
                  most important case
            \item specifying public shorthand variables in terms of other public variables,
                  using public vardef construct (the resulting public variables are therefore
                  visibly dependent on other public variables)
            \item specifying private shorthands (private variables defined in terms of
                  other private variables)
	    \end{itemize}
      \item independent (ghost) variables, which change only due to explicit abstract
            assignments
      \end{itemize}
      Most variables depend on the current state and are therefore \emph{non-pure}.
      Pure variables are just symbolic shorthands that define specification constants;
      note that specification constants can be higher-order.  Pure variables are defined
      using the \q{constdef} construct.
\item \emph{class invariants}, given by the $\q{invariant}$ keyword; class invariants
      can be 
      \begin{itemize}
      \item \emph{private}, describing properties of the encapsulated state of the class,
      or 
      \item \emph{public}, describing properties involving public variables of
      the class, which are either
          \begin{itemize}
          \item \emph{free} invariants, which talk only about public variables of the class 
            where they are declared, and are therefore guaranteed to hold by the class, or
          \item \emph{cross-class} invariants, which relate values of variables from multiple
            classes and place obligations on potentially the entire program
            (or an entire group of classes claimed by a common class)
          \end{itemize}
      \end{itemize}
\item $\q{readonly}$ variable modifier, indicating that a variable can
      only be modified from within a class; this is distinct from
      $\q{final}$ because variable can be modified even after initialization,
      as long as it is modified by the procedures within the module.
      This is analogous to the readonly \q{-} export in the programming language 
      Oberon-2 \cite[Page 127]{Nikitin98InRealmOberon}.
\item \emph{claimedby} construct for classes which establishes a simple, static, tree hierarchy
      of classes that allows public variables of 
      a class to be treated as a representation of another class
\item \emph{rep} ghost variable of each class, specifying the dynamicallly changing
      set of representation objects
      of a class
\end{enumerate}

In invariants within a procedure and in the ensures clause,
it is possible to refer to the value of a variable $x$ at
the entry of the procedure using the notation $\q{old}\, x$.

Specification constructs are written as annotations in Java
using special comments that start with \q{//:} and end as
normal comment signs.  Such annotations are automatically
ignored by Java compilers and the same source can be used
both as input to Jahob and as input to Java compiler.

See \q{src/form/yaccSpec.mly} for the precise syntax of the
specification constructs.

\subsection{Specification Language}

The properties of program state are specified using formulas
that involve sets and relations and can refer to the
variables of the state, including both concrete Java
variables and the speciification variables.

The concrete syntax of formula is given using the notation
based on formulas used in the Isabelle interactive theorem
prover \cite{NipkowETAL02IsabelleHOL}.
We only support basic operations on sets and relations,
see \q{src/form/yaccForm.mly} for the syntax of specification
constructs.

Quantification over objects: quantify over all objects that
existed and will exist.  Use T.alloc to denote currently
allocated objects of type T.  All objects exist in each state,
it is just T.alloc that changes, and only allocated objects
are reachable.

In procedure contracts, \q{result} denotes the resulting value of a
procedure.

In formulas, both the beginning and the end of a string
value are denoted by two openning apostrophe signs (\q{''})
and are thus not confused with quotes (\q{"}), which are
used to delimit the formulas.

