\section{More on Using Jahob Constructs}

\subsection{Claimed Fields}

A class $C$ can claim a field $f$ of another class $D$.  To specify
this relationship, write the annotation $\q{claimedby}\ C$ as a
modifier of the declaration of the field $f$.  

In this way, $f$ becomes part of the representation of $D$, which
means that only $D$ and the classes directly owned by $D$ can modify
$f$.

Ownership exists among classes as well.  Parents see children's public
variables, and children see their sibling's public variables.

If no owner class is specified, a special class World is assumed; thus
two classes with no owners are siblings.

Field ownership overrides class ownership, it's an exception to class
ownership.

\subsection{Modifies clauses.}
\label{sec:modifies}

Modifies clauses cannot refer to private or public shorthands.

Modifies clauses of public procedures in class $C$ can only refer to
non-shorthand public specification variables of $C$ and public
variables owned by owner of the current class.

Modifies clauses of private procedures in class $C$ can only refer
to non-shorthand private variables and public variables of ...

Top-level occurrence of \q{x..f} denotes \q{f} location of \q{x}
as opposed to the value denoted by \q{x..f}.

\subsection{Receiver Parameter}

On right hand side, if the name $f$ of a variable is a concrete
non-static field or a specfield of the current class, and it is not
qualified by some class name, and is not immediatelly enclosed in an
expression of the form $x..f$, then $f$ is implicitly replaced by
$this..f$.

When specifying a field name, a receveir parameter
\q{this} is understood to qualify this name.
On the left-hand side, a reference to a field \q{f} of the current
class \q{C} is replaced by \q{this..f}.  This rule applies to \q{requires} and
\q{ensures} clauses.  
Similarly, when using
\q{vardefs} to define the field \q{f}, the right-hand side $R$ denotes the
value of \q{this..f}, so the value of the entire field is
the union of $R$ where $\q{R}$ ranges over all objects of
the current class.

This desugaring also applies to modifies clauses, although
it there results in a different semantics.

\subsection{Modifies Clause Desugaring}

Assume that the modifies clause contains items of the one of the two
forms 1) \q{v} where \q{v} is a variable (or field) declared
in the class 2) \q{e..f} where \q{e} is an expression
evaluating to an object of the class that contains an \q{f}
field.  Using the desugaring, if \q{v} is a field of the current
class, then it is transformed into the second form as \q{this..v},
unless it is qualified with the class name.  Assume this transformation
has been done.

Let \q{v} be a variable in the static representation of the
class (potentially a field).  Then if \q{v} or \q{e..v} is
in modifies clause, it is removed from the list of variables
for which full frame condition is introduced.

If \q{f} is a field such that no item of the form \q{f} appears
in modifies clause, but an item of the form \q{e..f} appears
in the modifies clause, then let $\q{e}_1\q{..f}$,
\ldots, $\q{e}_n\q{..f}$ be all such expressions.  As just described,
then \q{f} is removed from the list of variables with full
frame conditions, but a partial frame condition is introduced of
the form:
\begin{verbatim}
ALL x::obj. x: old alloc.T & x ~= null & 
               x ~= old(e1) & ... & x ~= old(en) &
               x :~ rep
  -->  
    x..f = old (x..f)
\end{verbatim}
where \q{rep} is the variable denoting representation set of objects.

\subsection{Permissible Sets of Modifiers}
Cannot have claimed static variables, must be instance variables.  
If readonly, then public.

