\documentclass[preprint,nocopyrightspace]{sigplanconf}
%\usepackage[square, comma, sort&compress]{natbib}
\usepackage{proof}

\usepackage{amsmath}
\usepackage{color}

\newcommand{\todo}[1]{\textcolor{red}{#1}}

\newcommand{\cd}[1]{\texttt{#1}}

\newcommand{\coq}[1]{\mathsf{#1}}
\newcommand{\return}[1]{\coq{return}(#1)}
\newcommand{\bind}{\leftarrow}
\newcommand{\emp}{\mathbf{emp}}
\newcommand{\sep}{\ast}
\newcommand{\pts}[2]{#1 \mapsto #2}
\newcommand{\new}[1]{\coq{new}(#1)}
\newcommand{\free}[1]{\coq{free}(#1)}
\newcommand{\rd}[1]{!#1}
\newcommand{\wri}[2]{#1 := #2}
\newcommand{\himp}{\Rightarrow}

\begin{document}

%\conferenceinfo{PLDI �09}{ todo }

%\copyrightyear{2005}

%\copyrightdata{1-59593-056-6/05/0006}

\preprintfooter{DRAFT}

%\titlebanner{DRAFT}

\title{Effective Interactive Proofs for Higher-Order Imperative Programs}


\authorinfo{Double blind}
{ }

\maketitle

\begin{abstract}
  We present a new approach for constructing and verifying
  higher-order, imperative programs using the Coq proof assistant.
  Our approach is similar to the shallow embedding of Hoare Type
  Theory (HTT) that was used in the Ynot project.  However, compared
  to Ynot, our new approach significantly reduces the burden on
  the programmer.  For example, in both systems we have constructed
  fully verified imperative data structures, such as hash-tables with
  higher-order iterators, but the verification burden in the new
  system is reduced by at least an order of magnitude compared to the
  old Ynot system, by replacing manual proof with automation.  The core of
  the automation is a simplification procedure for implications in
  higher-order separation logic, with hooks that allow programmers to
  add domain-specific simplification rules.
  
  Compared to competing approaches to data structure verification, our
  system includes much less code that must be trusted; namely, about a
  hundred lines of Coq code defining a program logic.  All of our
  theorems and decision procedures have or build machine-checkable
  correctness proofs from first principles, removing opportunities for
  tool bugs to create faulty verifications.  We argue for the
  effectiveness of our infrastructure by verifying a number of data
  structures and comparing to similar efforts within other projects.
\end{abstract}

%\category{CR-number}{subcategory}{third-level}

%\terms
%term1, term2

%\keywords
%keyword1, keyword2

\section{Introduction (GREG)}
\begin{itemize}
\item Want to verify imperative programs
\item Higher-order functions
\item modularity
\item {\it mom and apple pie}
\end{itemize}

  Ynot is the right design from a modularity, re-use perspective.  Much of the success hinges on the use of a higher-order dependently-typed language (Coq) and the ability to smoothly integrate modeling (inductive definitions), domain-specific abstractions (e.g., STsep) and uniform abstraction (Pi). If Ynot is to succeed, need drastic improvements in automation.


\section{Overview (GREG)}
\begin{itemize}
\item Features of Ynot
\item problem with the proofs
\end{itemize}


\section{The Ynot Programming Environment}

The axiomatic base of Ynot is a fairly standard Hoare logic.  The main difference of our logic from usual presentations is that it is designed to integrate well with Coq's functional programming language, so that we formalize a language of expressions instead of commands.  A program derivation is of the form $\{P\} \; e \: \{Q\}$, where $P$ is a precondition predicate over heaps, and $Q$ is a postcondition predicate over initial heaps, functional values of $e$, and final heaps.  For instance, we can derive
$$\{\lambda \_. \; \top\} \; \return{1} \; \{\lambda h, v, h'. \; h' = h \land v = 1\}$$
and
$$\begin{array}{c}
  \{\lambda h. \coq{sel}(h, p_1) = p_2\} \\
  x \bind \; \rd{p_1}; \wri{x}{1} \\
  \{\lambda h, \_, h'. \; \coq{sel}(h, p_1) = p_2 \land h' = \coq{upd}(h, p_2, 1)\}
\end{array}$$

In actuality, Ynot has no separate concepts of programs and derivations.  Rather, the two are combined into one dependent type family, whose indices give the specification of a program.  For instance, the type of the first example program above would be:
\begin{verbatim}
ST (fun _ => True) (fun h v h' => h' = h /\ v = 1)
\end{verbatim}

Heaps are represented as functions from pointers to dynamically-typed packages, which are easy to implement in Coq with an inductive type definition.  The pointer read rule enforces that the heap value being read has the type that the code expects.  The original Ynot paper~\cite{ynot:icfp} contains further details of the base program logic.


\subsection{A Derived Separation Logic}

\begin{figure*}
  $$\infer{\{\emp\} \; \return{v} \; \{\lambda v'. \; [v = v']\}}{}
  \quad \infer{\{P_1\} \; x \bind e_1; e_2 \; \{Q_2\}}{
    \{P_1\} \; e_1 \; \{Q_1\}
    & (\forall x, \{P_2(x)\} \; e_2 \; \{Q_2\})
    & (\forall x, Q_1(x) \himp P_2(x))
  }$$

  $$\infer{\{\emp\} \; \new{v} \; \{\lambda p. \; \pts{p}{v}\}}{}
  \quad \infer{\{\exists v, \pts{p}{v}\} \; \free{p} \; \{\lambda \_. \; \emp\}}{}$$

  $$\infer{\{\exists v, \pts{p}{v} \sep P(v)\} \; \rd{p} \; \{\lambda v. \; \pts{p}{v} \sep P(v)\}}{}
  \quad \infer{\{\exists v, \pts{p}{v}\} \; \wri{p}{v'} \; \{\lambda \_. \; \pts{p}{v'}\}}{}$$

  $$\infer{\{P\} \; e \; \{Q\}}{
    P \himp P'
    & \{P'\} \; e \; \{Q'\}
    & Q' \himp Q
  }
  \quad \infer{\{P \sep R\} \; e \; \{Q \sep R\}}{
    \{P\} \; e \; \{Q\}
  }$$

  \caption{\label{STsep}The derived separation logic}
\end{figure*}

Direct reasoning about heaps leads to very cumbersome proof obligations, with many sub-proofs that pairs of pointers are not equal.  Separation logic~\cite{separation} is the standard tool for reducing that complexity.  The old Ynot system built a separation logic on top of the axiomatic foundation, and we do the same here.  As before, we introduce no new syntactic class of separation logic formulas.  Instead, we define functions that operate on arbitrary predicates over heaps, with the intention that we will only apply them on separation-style formulas.  Nonetheless, it can be helpful to think of our ``assertion language'' as defined by:
$$\begin{array}{rcl}
  P &::=& [p] \mid \emp \mid \pts{x}{y} \mid P \sep P \mid \exists x, P
\end{array}$$

For any pure Coq proposition $p$, $[p]$ is the heap predicate that asserts that $p$ is true and the heap is empty.  $\emp$ asserts that the heap is empty, and $\pts{x}{y}$ asserts that the heap contains only a mapping from $x$ to $y$.  $P_1 \sep P_2$ asserts that the heap can be broken into two heaps $h_1$ and $h_2$ with disjoint domains, such that $h_1$ satisfies $P_1$ and $h_2$ satisfies $P_2$.  Finally, we add existential quantification.

It is worth pointing out that we simplify the assertion language substantially by taking advantage of Coq's base language.  We do not need to include program variables in the logic, because any Coq variable may be included anywhere in a Coq development, including within one of our assertions.  The same argument allows us to avoid explicit mention of specification variables, sometimes also called ``ghost state'' variables.

We can write much more expressive formulas than in most systems based on separation logic.  Not only can any pure proposition be injected with $[\cdot]$, but we can also use arbitrary Coq computation to build impure assertions.  For instance, we can include calls to custom recursive functions that return assertions.  We need no special support in the assertion language to accommodate this, and Coq's theorem-proving support for reasoning about pattern-matching recursive functions can be brought to bear without modification.

This automatic importation of Coq features has some surprising and pleasant consequences.  We literally need no more of the standard separation logic connectives to verify a wide variety of data structures.  For instance, separation logic proofs often use several kinds of disjunction.  In our setting, we reduce this disjunction to pattern matching over datatypes.  We take advantage of Coq's computational rules for simplifying pattern matches, which need not mention heaps, making them easier to think about and promoting easier proof automation.

We do need a notion of implication, though we do not need to include it in assertions.  There is no need to introduce the ``magic wand'' of separation logic.  Rather, we just define an almost trivial operator $\himp$, which appears only at the top level of individual proof obligations.  The meaning of $p \himp q$ is that any heap satisfying $p$ also satisfies $q$.  The relevant aspects of the usual ``magic wand'' connective are built into the definition of what a valid program derivation is in the separation logic.

\medskip

What we have described so far is the same as in the original Ynot work, with the exception that that work used some additional separation connectives that we no longer need.  The big departure of our new system is that we define a more standard separation logic.  The old Ynot's separation logic included ``binary postcondtions'' that may refer to both the initial and final heaps.  This is in stark contrast to traditional separation logics, where all assertions are simple separation formulas, and all verification proof obligations are simple implications between such assertions.  The utility of that formalism has been born out in the wealth of tools that have used separation logic for automated verification.  In contrast, verifications with the old Ynot tended to involve at least tens of steps of manual proof per line of program code.

Why did the original Ynot use this nonstandard program logic?  The answer has to do with the need for an effective analogue of specification variables.  In traditional separation logic, specification variables are commonly used to ensure that parts of state are preserved by commands, when the same specification variable appears in both the precondition and postcondition of a derivation.  In contrast, the old Ynot used binary postconditions for the same purpose, asserting equations between parts of the pre and post heaps.

The final output of a Ynot program is an executable piece of OCaml or Haskell code, produced with Coq's program extraction facility.  We can use standard Coq variables as specification variables, but, by default, they will remain at runtime, reducing the efficiency of programs.  One of the main innovations of the new work we present involves a new way of encoding specification variables in Coq, such that they will be erased by the standard program extraction facility.  This makes it possible to use a more standard separation logic, with the corresponding decrease in the difficulty of automating proofs.  We will present this new technique by example in later subsections.

Figure \ref{STsep} presents the main rules of our separation logic.  The notable divergence from common formulations is in the use of existential quantifiers in the rules for freeing, reading, and writing.  These differences make sense because Ynot is implemented within a constructive logic.  A more standard, ``classical'' separation logic would, for instance, require that, in the rule for $\coq{free}$, the value $v$ pointed to by $p$ be provided as an argument to the proof rule.  In constructive logic, such a value can only be produced when it can be computed by an algorithm.  Not only that, but we would not be able to use any facts implied by the current heap assertion to build one of these witnesses, and perhaps the witness can only be proved to exist using such facts.  The explicit existential quantifier frees us to reason ``inside the assertion language'' in finding the witness.

One consequence is that the ``read'' rule must take a kind of explicit framing condition.  This condition is parameterized by the value being read from the heap, making it a kind of description of the ``neighborhood around that value'' in the heap.  More standard separation logics force the exact value being read to be presented as an argument to the proof rule, but here we want to allow verification of programs where the exact value to read cannot be computed from the pieces of pure data that are in scope.

\medskip

In the rest of this section, we will introduce the Ynot programming environment more concretely, via several examples.


\subsection{Verifying an Implementation of Imperative Stacks}

\begin{figure}
  \begin{verbatim}
Module Type STACK.
  Parameter t : Set -> Set.
  Parameter rep T : t T -> list T -> hprop.

  Parameter new T :
    STsep __ (fun s : t T => rep s nil).
  Parameter free T (s : t T) :
    STsep (rep s nil) (fun _ : unit => __).

  Parameter push T (s : t T) (x : T) (ls : [list T]) :
    STsep (ls ~~ rep s ls)
          (fun _ : unit => ls ~~ rep s (x :: ls)).
  Parameter pop T (s : t T) (ls : [list T]) :
    STsep (ls ~~ rep s ls) (fun xo : option T => ls ~~
            match xo with
              | None => [ls = nil] * rep s ls
              | Some x => Exists ls' :@ list T,
                            [ls = x :: ls'] * rep s ls'
            end).
End STACK.
  \end{verbatim}

  \caption{\label{Stack}The signature of an imperative stack module}
\end{figure}

Figure \ref{Stack} shows the signature of a Ynot implementation of the stack ADT.  The signature is expressed in Coq's ML-like module system.  Each implementation contains a type family \cd{t}, where, for any type \cd{T}, a value of \cd{t(T)} represents a stack storing values of \cd{T}.  An abstract type in ML can be thought of as ``standing for the underlying invariant.''  For Ynot, this is true for ``pure'' invariants only.  We need to add an additional ``impure'' invariant explicitly.  The type \cd{hprop} stands for assertions, and the parameter \cd{rep} is an assertion-valued function over a stack and a \emph{functional model} of the stack.  Each stack can be thought of as standing for a particular pure list, and the \cd{rep} predicate formalizes this connection.

The type of the \cd{new} operation tells us that it expects an empty heap on input, and on output the heap contains just whatever mappings are needed to satisfy the representation invariant between the function return value and the empty list.  The ASCII notation \cd{\_\_} stands for the $\emp$ of usual pencil-and-paper separation logics.  The \cd{free} operation takes a stack \cd{s} as an argument, and it expects the heap to satisfy \cd{rep} on \cd{s} and the empty list.  The post state shows that all heap values associated with \cd{s} are freed.

The specification for \cd{push} says that it expects any valid stack as input and modifies the heap so that the same stack that stood for some list \cd{l} beforehand now stands for the list \cd{x :: l}, where \cd{x} is the appropriate function argument.  We see an argument \cd{ls} with type \cd{[list T]}.  The brackets are a notation defined by the Ynot library, standing for \emph{computational irrelevance}.  That is, the type-checker should enforce that the value of \cd{ls} is not needed to execute the function.  Rather, such values may only be used in stating specifications and discharging proof obligations.  In other words, irrelevant variables act just like specification variables, but we do not need to build any special support for them into our assertion language.  We use Coq's notation scope mechanism to overload brackets for writing irrelevant types and lifted pure propositions.

For an assertion \cd{P} that mentions the irrelevant variable \cd{x}, the notation \cd{x \textasciitilde\textasciitilde \; P} must be used to explicitly ``unpack'' \cd{x}.  The type of the unpack operation is such that it may only be applied to assertions and may not be used to allow an irrelevant variable's value to ``leak into'' the computational part of a program.

The type of \cd{pop} showcases how we avoid the disjunctive connectives of separation logic.  The function returns an optional \cd{T} value, which will be \cd{None} when the stack is empty and will be \cd{Some x} when \cd{x} is at the top of the stack.  We use a Coq \cd{match} expression to give a different postcondition for each case.

\medskip

We can implement a module satisfying this signature.  With the type \cd{T} as a local variable, we can define the type of nodes of the linked lists that we will use.

\begin{verbatim}
Record node : Set := Node {
  data : T;
  next : option ptr
}.
\end{verbatim}

To define the representation invariant, we want a recursive function specifying what it means for a possibly-null pointer to represent a functional list.

\begin{verbatim}
Fixpoint listRep (ls : list T) (hd : option ptr)
    {struct ls} : hprop :=
  match ls with
    | nil => [hd = None]
    | h :: t => match hd with
                  | None => [False]
                  | Some hd => Exists p :@ option ptr,
                      hd --> Node h p * listRep t p
                end
  end.
\end{verbatim}

We can represent stacks as untyped pointers to the heads of linked lists built from \cd{Node}s.

\begin{verbatim}
Definition stack := ptr.
\end{verbatim}

We achieve type safety through the representation invariant.

\begin{verbatim}
Definition rep (s : stack) (ls : list T) : hprop :=
  Exists po :@ option ptr, s --> po * listRep ls po.
\end{verbatim}

Before we start implementing the ADT methods, we should set up some proof automation machinery.  Systems like Smallfoot~\cite{smallfoot} have hardcoded support for particular heap predicates like acyclic linked list-ness, cyclic linked list-ness, and so on.  These systems perform ad-hoc simplifications on formulas that mention the predicates that they understand.  In contrast, with Ynot, the programmer can define his own new predicates, as we have just done.  Not only that, but he can also prove lemmas that correspond with the simplification rules built into automated tools, and he can plug his lemmas into a general separation logic solver.  All of this is done with no risk that a mistake by the programmer will lead to a faulty verification; every lemma must be proved from first principles.

For the current example, we need two lemmas for ``unfolding'' the definition of \cd{listRep} at points where we know for sure whether or not the pointer argument is null.

\begin{verbatim}
Theorem listRep_None : forall ls,
  listRep ls None ==> [ls = nil].
  destruct ls; sep fail idtac.
Qed.

Theorem listRep_Some : forall ls hd,
  listRep ls (Some hd) ==> Exists h :@ T,
    Exists t :@ list T, Exists p :@ option ptr,
      [ls = h :: t] * hd --> Node h p * listRep t p.
  destruct ls; sep fail ltac:(try discriminate).
Qed.
\end{verbatim}

The proofs are given in Coq's tactic language.  We ask to do a case analysis on the structure of the list \cd{ls}, applying the parametrized separation logic solver in each case.  We will go into more detail shortly on the significance of the two arguments to the \cd{sep} procedure.

Now we can plug our two unfolding rules into the separation solver.  We define a local procedure for simplifying separation implications.

\begin{verbatim}
Ltac simp_prem :=
  simpl_IfNull;
  simpl_prem ltac:(apply listRep_None
                   || apply listRep_Some).
\end{verbatim}

Our procedure, or \emph{tactic} in Coq parlance, first calls a simplification procedure associated with a syntax extension for checking pointer nullness.  Next, our procedure calls a tactic \cd{simpl\_prem} from the Ynot library, for simplifying premises of implications.  The argument to \cd{simpl\_prem} gives a procedure to attempt on each premise, until no further progress can be made.

We can plug our domain-specific simplification tactic into the generic separation logic solver.  We define a new tactic that will suffice to prove all goals that will arise in this example.  The first argument to \cd{sep} gives a procedure to apply to simplify premises before starting the main proof search, and the second argument gives a procedure to attempt on individual subgoals throughout proof search.  The \cd{discriminate} tactic solves goals whose premises include inconsistent equalities over values of datatypes, like \cd{nil = x :: ls}; and adding \cd{try} in front prevents \cd{discriminate} from signaling an error if no such equality exists.

\begin{verbatim}
Ltac t := unfold rep; sep simp_prem
                          ltac:(try discriminate).
\end{verbatim}

We implement each method by stating its type as a proof search goal, using tactics to realize the goal step by step.  Here is the implementation of the \cd{new} operation.

\begin{verbatim}
Definition new : STsep __ (fun s => rep s nil).
  refine {{New (@None ptr)}}; t.
Qed.
\end{verbatim}

A simple two-step proof script suffices.  We first use the \cd{refine} tactic to provide a ``template'' for the implementation.  The template may have holes in it, and each hole is added as a subgoal.  We chain our \cd{t} tactic with the semicolon operator, so that \cd{t} is applied to each subgoal generated from a hole.  This is enough to finish the implementation and proof.

The holes in the refinement are not apparent from the syntax we have used.  We apply Coq's syntax extension system to hide such details for most programs.  In this case, it is the \cd{\{\{...\}\}} syntax that hides the holes.  This syntax requests simultaneous strengthening and weakening.  It expands to an explicit call to a rule that takes two proofs as arguments.  Those arguments are filled in as holes by the syntax extension.

The definitions of \cd{free} and \cd{push} are not much more complicated.

\begin{verbatim}
Definition free s : STsep (rep s nil)
    (fun _ : unit => __).
  intros; refine {{Free s}}; t.
Qed.
\end{verbatim}

\begin{verbatim}
Definition push s x ls : STsep (ls ~~ rep s ls)
    (fun _ : unit => ls ~~ rep s (x :: ls)).
  intros; refine (hd <- !s;
    nd <- New (Node x hd);
    {{s ::= Some nd}}
  ); t.
Qed.
\end{verbatim}

The implementation of \cd{pop} uses another syntax extension, which provides an \cd{IfNull} expression form.  The \cd{option}-typed argument to \cd{IfNull} is checked for nullness (i.e., equality to \cd{None}).  In an \cd{Else} branch, where the pointer is known to be non-null, that fact is added as a usable proof hypothesis, and the variable being tested is rebound with a non-\cd{option} type.

\begin{verbatim}
Definition pop s ls :
  STsep (ls ~~ rep s ls) (fun xo => ls ~~
          match xo with
            | None => [ls = nil] * rep s ls
            | Some x => Exists ls' :@ list T,
                          [ls = x :: ls'] * rep s ls'
          end).
  intros; refine (hd <- !s;
    IfNull hd Then
      {{Return None}}
    Else
      nd <- !hd;
      Free hd;;
      s ::= next nd;;
      {{Return (Some (data nd))}}); t.
Qed.
\end{verbatim}

We complete the implementation with a trivial definition of the type family \cd{t}, relying on the representation invariant to ensure proper use.

\begin{verbatim}
Definition t (_ : Set) := stack.
\end{verbatim}

\begin{figure}
  \begin{verbatim}
module Stack = 
 struct 
  type 't node = { data : 't; next : ptr option }
  let data n = n.data
  let next n = n.next
  type stack = ptr
  
  let coq_new =
    sepWeaken (sepStrengthen (sepFrame (sepNew None)))
  
  let free s =
    sepWeaken (sepStrengthen (sepFrame (sepFree s)))
  
  let push s x =
    sepBind (sepStrengthen (sepRead s)) (fun hd ->
      sepBind (sepStrengthen (sepFrame
          (sepNew { data = x; next = hd })))
        (fun nd ->
        sepWeaken (sepStrengthen (sepFrame
          (sepWrite s (Some nd))))))
  
  let pop s =
    sepBind (sepStrengthen (sepRead s)) (fun hd ->
      match hd with
        | Some v ->
            sepBind (sepStrengthen (sepRead v)) (fun nd ->
              sepSeq (sepStrengthen (sepFrame (sepFree v)))
                (sepSeq (sepStrengthen (sepFrame
                    (sepWrite s (next nd))))
                  (sepWeaken
                    (sepStrengthen (sepFrame
                      (sepReturn (Some (data nd))))))))
        | None -> sepWeaken (sepStrengthen (sepFrame
                    (sepReturn None))))
  
  type 'x t = stack
 end
  \end{verbatim}

  \caption{\label{extracted}OCaml code extracted from the stack example}
\end{figure}

For our modest efforts, we can now extract an executable OCaml version of our module.  Figure \ref{extracted} shows the result of running Coq's automatic extraction command on our \cd{Stack} module.

In the method implementations, we see invocations of functions whose names begin with \cd{sep}.  These come from the Ynot library, and we must provide their OCaml implementations.  Any Ynot program that returns a type \cd{t} may be represented in \cd{unit -> t} in OCaml, regardless of the specification appearing in the original Coq type.  This makes it easy to implement the basic functions, in the spirit of how the Haskell IO monad is implemented.  We see calls to explicit weakening, strengthening, and framing rules in the extracted code.  In OCaml, these can be implemented as no-ops and elided by an optimizer.

Notice that all specification variables and proofs are eliminated automatically by the Coq extractor.  With the elision of weakening and related operations, we arrive at exactly the kind of monadic code that is standard fare for Haskell, such that the body of optimizations developed for Haskell can be put to immediate use in creating an efficient compilation pipeline for Ynot.


\subsection{Verifying Imperative Queues}

It is not much harder to implement and verify a queue structure.  We define an alternate list representation, parameterized by head and tail pointers.

\begin{verbatim}
Fixpoint listRep (ls : list T) (hd tl : ptr)
    {struct ls} : hprop :=
  match ls with
    | nil => [hd = tl]
    | h :: t => Exists p :@ ptr, hd --> Node h (Some p)
                  * listRep t p tl
  end.

Record queue : Set := Queue {
  front : ptr;
  back : ptr
}.

Definition rep' (ls : list T) (fr ba : option ptr) :=
  match fr, ba with
    | None, None => [ls = nil]
    | Some fr, Some ba => Exists ls' :@ list T,
        Exists x :@ T, [ls = ls' ++ x :: nil]
          * listRep ls' fr ba * ba --> Node x None
    | _, _ => [False]
  end.
          
Definition rep (q : queue) (ls : list T) :=
  Exists fr :@ option ptr, Exists ba :@ option ptr,
    front q --> fr * back q --> ba * rep' ls fr ba.
\end{verbatim}

For this representation, we prove similar ``unfolding'' lemmas to those we proved for stacks, with comparable effort.  We also need a new lemma for ``unfolding a queue from the back.''

\begin{verbatim}
Lemma rep'_back : forall ls fr ba,
  rep' ls (Some fr) ba
  ==> Exists nd :@ node, fr --> nd
    * Exists ls' :@ list T, [ls = data nd :: ls']
      * match next nd with
          | None => [ls' = nil]
          | Some fr' => rep' ls' (Some fr') ba
        end.
\end{verbatim}

The proof of the lemma relies on some lemmas about pure functional lists.  With those available, we prove \cd{rep'\_back} in under 20 lines.  When we plug this and the two other unfolding lemmas into the \cd{sep} procedure, we arrive at quite a robust decision procedure for separation assertions about lists that may be modified at either end.

Again, every proof obligation for our queue implementation is proved by a \cd{t} tactic built from \cd{sep}.  We write under 10 lines of new tactic hints to be applied during proof search, and we must prove one key lemma by induction.

\begin{verbatim}
Lemma push_listRep : forall ba x nd ls fr,
  ba --> Node x (Some nd) * listRep ls fr ba
  ==> listRep (ls ++ x :: nil) fr nd.
  Hint Resolve himp_comm_prem.
  induction ls; t.
Qed.
\end{verbatim}

We suggest a hint based on commutativity of separating conjunction, state the inductive structure of the proof, and discharge the two cases with our specialized \cd{t}.  This is an instance of a common pattern, where lemmas requiring inductive proofs must be stated explicitly, while most other separation facts are proved automatically by \cd{sep}.


\subsection{Arrays}

It is also easy to support arrays within our framework.  We can define suitable constants in the base program logic:

\begin{verbatim}
Parameter array : Set.
Parameter array_length : array -> nat.  
Parameter array_plus : array -> nat -> [ptr].
\end{verbatim}

The \cd{array\_plus} function is for computing a pointer to a particular cell of an array.  Note how the type of this function forces the resulting pointer to be treated in a computationally irrelevant way.  That is, it is easy to reason about the separate cells of an array with the separation logic connectives we already introduced, but client code cannot actually calculate pointers from arrays ``at runtime.''  This makes it possible to represent Ynot arrays with OCaml arrays.

With an infix notation \cd{@+} defined for \cd{array\_plus}, it is easy to write a type for array update.  We use the notation \cd{x :\textasciitilde\textasciitilde v in P} to bind \cd{x} to the hidden value of specification value \cd{v} within the assertion \cd{P}.  We use the shorthand \cd{p -->?} to indicate that \cd{p} points to some unknown value of unknown type.

\begin{verbatim}
Parameter upd_array (A:Type)(a:array)(i:nat)(v:A) :
  STsep (p :~~ a @+ i in p -->?)
        (fun _ : unit =>
               p :~~ a @+ i in p --> v).
\end{verbatim}

The type of the array allocation operation indicates that every pointer within the array is initialized to some unknown value.  The notation \cd{\{@ e | i <- min + len\}} is inspired by standard set comprehension notations, and it denotes an iterated separating conjunction where the variable \cd{i} ranges from \cd{min} to \cd{min + len - 1}.

\begin{verbatim}
Parameter new_array (n:nat) :
  STsep __ (fun a : array => [array_length a = n]
    * {@ p :~~ a @+ i in p -->? | i <- 0 + n}.
\end{verbatim}


\subsection{Loops}

\begin{verbatim}
Definition getElements (hd : option ptr) (ls: [list T]) :
  STsep (ls ~~ listRep ls hd)
        (fun res : list T =>
          ls ~~ [res = ls] * listRep ls hd).
  refine (Fix2
    (fun hd ls => ls ~~ listRep ls hd)
    (fun hd ls res => ls ~~ [res = ls] * listRep ls hd)
    (fun self hd ls => 
      IfNull hd Then
        {{Return nil}}
      Else
        fn <- !hd;
        Assert ((ls ~~ [head ls = Some (data fn)]
            * hd --> fn)
          * (ls' :~~ (ls ~~~ tail ls) in
              listRep ls' (next fn)));;
        rest <- self (next fn) (ls ~~~ tail ls) <@> _;
        {{Return (data fn :: rest)}})); t'.
Qed.
\end{verbatim}


%% \section{Comparison to Previous Ynot (AVI)}

%%  The original formulation of Ynot lead to large, unwieldy proofs.
%%  Reasoning about separation logic connectives was done at the heap
%%  level; a larger fraction of all of our proof scripts simply massaged
%%  subeaps into appropriate forms.  We had some tactics that implemented
%%  basic operations, such as the split, join, and flattening of
%%  sub-heaps, but the proof writer was forced to stitch them together to
%%  force the heaps into the correct form.

%%  The focus of our re-implementation of the Ynot system has been proof
%%  automation.  Throught the use of the sep tactic, we have virtually
%%  eliminated the need to reason about heaps; reasoning is done at the
%%  level of seperation logic.  Much of this reasoning has in turn been
%%  automated by a set of tactics such as \texttt{sep auto}. These
%%  tactics reduce the need to reason about the connectives of separation
%%  logic and their properties, allowing the proof writer to focus on the
%%  domain specific parts of the proof.  For proofs that only have simple
%%  domain specific parts, \texttt{sep auto} is often able to prove them
%%  on its own.

%%  The new model uses ``ghost variables'' to simplify specifying
%%  properties. It does so in a way that guarantees that they are
%%  computationally irrelevant, and so can always be eliminated when a
%%  program is extracted.  This in turn allows the new model to use unary
%%  post-conditions, which greatly simplifies specifications and enables
%%  the greater automation.  The old model used binary post-conditions to
%%  obviate the need for ghost variables.  Binary post-conditions where
%%  generally harder to reason about, since the connection between pre-
%%  and post-conditions always needed to be re-established.  In the new
%%  style, the connection is built in to the specification.

%%  In the new Ynot, proofs are an order of magnitude shorter then they
%%  where in the old Ynot. (provide numbers here)

\section{Evaluation}
\begin{figure*}
\begin{center}
\begin{tabular}{r | r | r | r | r | r | r | r}
 & lines of code & total  non-code & annotations & lemmas & proofs & aux. def & tactics\\\hline
{\tt LL-ref} & AVI\\
{\tt LL-Jahob} & RYAN\\
{\tt LL-Smallfoot} & GREGORY\\
{\tt Hashtable} & AVI\\
{\tt Binary Heap} & RYAN\\
{\tt BST} & ??\\
{\tt Stack} & ADAM\\
{\tt Queue} & ADAM\\
\end{tabular}
\end{center}
\end{figure*}

\section{Related Work}
\subsection{Jahob (RYAN)}

The Jahob system~\cite{jahob} allows the specification and verification 
of recursive, linked data structures in a fragment of Java.  Like ynot, 
verified programs are correct up to termination.  

Jahob specifications are written in classical higher order logic.  
The abstract state of an object is given by s a collection of 
programmer defined ``specvars'' which do not exist during program execution.
Specvars are defined by Jahob formulae, which resemble formulae of set 
theory in Isabelle/HOL.  Jahob formulae are simply typed with ground types
bool, obj, and int, type constructors $\Rightarrow$ for total functions, * for tuples
and set for sets.  The logic contains polymorphic equality, standard
logical connectives $\wedge, \vee, \neg, \to, \forall, \exists$ and
$\lambda$ binders, set comprehension, operations on sets ($\cup, \in$) 
and transitive closure, finite set cardinality, and a tree function for indicating
that a structure is a tree.  Here is the interface for a Jahob association list: 

\begin{verbatim}
class AssocList {
//: public specvar content :: "(obj * obj) set
public Object put(Object k0, Obkect v0)
/*: requires "k0 <> null /\ v0 <> null"
    modifies content
    ensures  
    "content = old content 
               - {(k0, result)} U {(k0, v0)} 
     /\ (result =  null -> 
          ~ exists v, 
             (k0, v) in old content) /\
     /\ (result <> null -> 
           (k0, result) in old content) */
{...}
public Object get(Object k0)
/*: requires "k0 <> null"
    ensures  
    "result =  null -> 
      ~ exists v, (k0, v) in content 
      /\ result <> null -> 
           (k0, result) in content" */
}
\end{verbatim}

Like ynot, Jahob relates the abstract state of a data structure to its concrete heap representation.  
One way to implement the association list with a singly linked list is to represent links between 
nodes using an edge relation (vardef is a shorthand definition):

\begin{verbatim}
public /*: claimedBy AssocList */ 
class Node {
    public Object key; public Object value; 
    public Node next;
    //: public ghost specvar cnt 
         :: "(obj * obj) set" = "{}"
}
private static specvar edge 
  :: ``obj => obj => bool'';

vardefs ``edge == (fun x y => 
  (x in Node /\ y = x..next) \/
  (x in AssocList /\ y = x..first))'';
invariant InjInv: forall x1 x2 y,
   y <> null /\ edge x1 y 
   /\ edge x2 y -> x1 = x2'';
\end{verbatim} 

The (recursive) representation invariant defines the members of the abstract set 
representing the linked list as the union of the element in the head node 
and the members of the subsequent nodes.  It also ensures that keys only occur once in the list.

\begin{verbatim}
private Node first;

vardefs "content == first..cnt";

invariant CntDef:
    ``forall x, x in Node /\ x in alloc 
        /\ x <> null ->
           x..cnt = {(x..key, x..value)} 
              U x..next..cnt /\
           (forall v, (x..key, v) 
              notin x..next..cnd)'';
invariant CntNull:
    ``forall x, x in Node /\ x in Alloc 
        /\ x = null -> x..cnt = {}'';
\end{verbatim}

Ynot's approach to specification and abstract state is similar in spirit to Jahob's but 
feels more computational in practice.
Because ynot is embedded in Coq, specifications and programs share a common language.
This makes it possible to define specifications using programs with reduction behavior
rather than with formulae of set theory; reasoning about such computational entities
is at the core of ynot's proof automation.  In addition, ynot uses separation connectives
to describe the heap.  For instance, the representation invariant rep for the ynot
singly linked list implementation is defined as a Coq fixpoint of essentially a separation logic formula:

\begin{verbatim}
Variables K V: Set.
Variable eqK : forall (k1 k2: K), {k1=k2} + {k1<>k2}.

Record Node : Set := node {
  key: K; value: V; next: option ptr
}.

Definition AssocList : Set := ptr.

Fixpoint rep' (m : list (prod K V)) 
(p : option ptr) {struct m} : hprop :=
  match p with
    | None => [m = nil]
    | Some hd => 
      match m with
        | (k,v) :: b => Exists nxt :@ option ptr, 
             hd --> node k v nxt * rep' b nxt
        | nil => [False]
      end
  end.

Definition rep (m: list (prod K V)) 
 (ll: AssocList) : hprop :=
   Exists n :@ option ptr, 
    ll --> n * rep' m n.
\end{verbatim}

The computational behavior of specification means, for instance, that the type of new
\begin{verbatim}
Definition new : STsep __ (rep nil).
\end{verbatim}

is definitionally equal to 
\begin{verbatim}
  STsep __ (fun p => match p with 
                       | None => [nil = nil] 
                       | Some hd => [False]
                     end)
\end{verbatim}

and so it is easy to automate (using a tactic called t) the reasoning that proves
the new association list allocator is correct:
\begin{verbatim}
Definition new : STsep __ (rep nil).
  refine {{ New None }}; t. Qed.
\end{verbatim}

The specifications for put and lookup exploit other pure functional programs:

\begin{verbatim}
Fixpoint lookup (k: K) 
  (l: list (prod K V)) : option V :=
 match l with
  | nil => None
  | (k', v)::b => if eqK k k' 
                  then Some v 
                  else lookup k b
 end.

todo - change put 
Definition put (ll : AssocList) 
 (m : [list (prod K V)]) (k : K) (v : V) :
  STsep (m ~~ rep m ll)
        (fun _:unit => m ~~ rep ((k,v)::m) ll ).

Definition get'' (k: K) (hd: option ptr) 
 (m: [list (prod K V)]) : 
  STsep (m ~~ rep' m hd)
        (fun res:option V => m ~~ 
           [res = lookup k m] * rep' m hd).
\end{verbatim} 

The actual implementations of get in ynot and jahob are essentially identical.
In both cases annotations are required to guide the proving process.  (Add numbers here).
In Jahob:

\begin{verbatim}
/*: requires ``k0 <> null''
    ensures 
      ``(result <> null -> 
          (k0, result) in content 
          /\ (result =  null -> 
                ~exists v0, 
                  (k0, v) in content)) */
{
  Node current = first;
  while //: inv ``forall v, ((k0, v) in content) 
                    = ((k0, v) in current..cnt)''
   (current != null) {
     if (current.key == k0) { 
       return current.value; 
     }
     current = current.next
  }
  return null;
}
\end{verbatim}

In ynot:

\begin{verbatim}
Definition get (k: K) 
 (hd: option ptr) (m: [list (prod K V)]):
  STsep (m ~~ rep' m hd)
        (fun res:option V => m ~~ 
           [res = lookup k m] * rep' m hd).
intro k.
refine (Fix2
    (fun hd m => m ~~ rep' m hd)
    (fun hd m o => m ~~ 
       [o = lookup k m] * rep' m hd)
    (fun self hd m =>
      IfNull hd
      Then {{ Return None }}
      Else fn <- hd !! fun fn => m ~~ 
             [head m = Some (key fn, value fn)] 
             * rep' (tail m) (next fn);
           if eqK (key fn) k
           then {{ Return (Some (value fn)) }}
           else {{ 
            self (next fn) (m ~~~ tail m) 
              <@> (m ~~ hd --> fn * 
                  [head m = Some (key fn, value fn)]) 
           }}
      )); try solve [ t | hdestruct m; t ].
Defined.

Definition get (k: K) 
 (ll: AssocList) (m: [list (prod K V)]) :
   STsep (m ~~ rep m ll)
         (fun r:option V => m ~~ 
            rep m ll * [r = lookup k m]).
intros; refine (hd <- !ll;
    Assert (ll --> hd * (m ~~ rep' m hd));;
    {{get'' k hd m <@> _}});
  t.
\end{verbatim}

Whereas ynot's automated reasoning is completely captured by Coq tactics,
Jahob programs are verified using a series of stages similar to 
a compiler pipeline.  First, Jahob programs are translated into 
an extended guarded command language.
Then, Jahob generates verification conditions which are discharged 
using a combination of theorem provers -- if one prover does not succeed 
on an obligation, another is tried.  This allows Jahob to run multiple
provers in parallel.  Because provers are often specialized to particular
classes of formulae, formula approximation techiques must be used to integrate them with
a system using higher order logic. Jahob supports a number of 1st order and SMT provers, and also the 
MONA prover for the monadic fragment of 2nd order logic and Isabelle and Coq. 

 \subsection{Smallfoot (GREGORY)}
We conclude our comparison with alternative approaches by comparing
our system to Smallfoot~\cite{smallfoot}. The focus of Smallfoot
is on compltely automatic reasoning, and it's almost complete nature
makes many correct programs go through with only the function level
annotations, a goal, but not yet a complete reality in Ynot. In
addition to the focus on automatic proofs, Smallfoot includes simple
concurrency primitives, $||$ and $\mathsf{with}\,r\,\mathsf{when}(B)
C$ with which they are able to implement a producer-consumer
model. Based on these points, we will draw a comparison in two parts:
the size of annotations and proof obligations and the expressivity of
the logic.

On the size of proof obligations, the Smallfoot system comes out
ahead, requiring no user-constructed proofs. The drawback to this is
that only inductive data of the following types are allowed: singly
linked lists, doubly linked lists, trees and xor lists. This
restriction is necessary to Smallfoot in order to make the proofs
decidable.

The annotations in the example programs are small mainly because they
use these. While the predicates are hard-coded in Smallfoot,
predicates can be coded directly by the user in Ynot. The style of
Ynot representations are to express data by mapping a ``ghost''
representation into a heap representation. This approach allows
stronger dependent types in the contracts for functions, requireing
the accompanying proof to be a proof of partial correctness rather
than simply of resource constraints. For example, consider the
function to test whether a list is empty in Smallfoot and in Ynot.
\begin{verbatim}
is_empty(r;l) [l |-> t * list(t)] {
  local p;
  p = l->p;
  if (p == NULL) { r = 1; } 
  else { r = 0; }
} [l|-> t * list(t)]
\end{verbatim}

The following function also has the same type:
\begin{verbatim}
is_empty(r;l) [l |-> t * list(t)] {
  r = 1;
} [l|-> t * list(t)]
\end{verbatim}
In Ynot, the \verb|is_empty| function can be defined as follows:
\begin{verbatim}
Definition is_empty A (ll : LinkedList)
  (m : [list A]) :
  STsep (m ~~ rep m ll) 
    (fun res:bool => m ~~ match res with
                            | nil => [res = true]
                            | _   => [res = false]
                          end).
  intros;
  refine (hd <- !ll;
          IfNull hd Then {{Return true}}
          Else {{Return false}}); sep simpl.
Qed.          
\end{verbatim}
The type guarantees that the function is implemented correctly with
respect to the underlying list.

\todo{This is analagous to the implementation of llseg presented in
  the smallfoot paper.}
For example, the following fixpoint expresses the recursive definition
of a linked list segement.  \todo{What is the best way to encode a
  linked list segment?}
\begin{verbatim}
Fixpoint llseg A (hd tl : option ptr) 
  (m : list A) {struct m} :=
  match m with 
    | nil => [hd = tl]
    | a :: b => 
      match hd with 
        | None => [False]
        | Some p => [hd <> tl] * 
            Exists nxt :@ option ptr, 
              hd --> node a nxt * llseg nxt tl b
      end
  end.
\end{verbatim}




Comparison points
\begin{itemize}
\item The use of Coq allows quantifiers in formula. (makes inference
  more difficult but allows more expressive types)
\item Ynot is capble of arbitrary inductive predicates
\item Ynot does not support concurrency which is a strength of
  smallfoot (race detection)
\item Smallfoot is very fast
\item A lot of the ``parallel examples'' don't actually use
  synchronization. These would be pretty trivial to implement in Ynot
  with a function:
\begin{verbatim}
SepParallel (r1 r2 : Type) (P_i Q_i : hprop)
  (P_e : r1 -> hprop) (Q_e : r2 -> hprop) :
     STsep (P_i) (P_e) -> STsep (Q_i) (Q_e)
  -> STsep (P_i * Q_i) (fun r:(r1 * r2) => 
               (P_e (fst r)) * (Q_e (snd r)))
\end{verbatim}
the more interesting aspect of the parallelism comes from the locking
mechanism, which could be encoded in Ynot but only after a little bit
of hacking. (smallfoot uses resource invariants to control this)
\end{itemize}

Smallfoot examples:
\begin{itemize}
\item circular list
\item doubly linked list
\item merge sort
\item ``memory manager'' (basically just a stack)
\item queue
\item tree
\item xor linked list segment
\item xdeq --- uses parallelism (reading and writing buffer)
\end{itemize}

\section{Other related work}


\section{Conclusions \& Future Work}
Irrelevance, concurrency, IO/effects

automation/annotations (loop invariant inference)

Check that bibliography is working~\cite{htt}.

%\bibliographystyle{plainnat}
\bibliographystyle{plain}

\bibliography{bib}

\end{document}
